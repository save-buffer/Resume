\documentclass[letterpaper,11pt]{article}

\usepackage{latexsym}
\usepackage[empty]{fullpage}
\usepackage{titlesec}
\usepackage{marvosym}
\usepackage[usenames,dvipsnames]{color}
\usepackage{verbatim}
\usepackage{enumitem}
\usepackage{hyperref}
\usepackage{fancyhdr}
\usepackage{tabularx}
\usepackage{fontawesome}
\usepackage{tikz}
\usepackage{graphicx}
\usepackage[most]{tcolorbox}
\usepackage{paracol}

\usepackage{microtype}

\usepackage[sfdefault, thin]{roboto}
\renewcommand\familydefault{\sfdefault} 
\usepackage[T1]{fontenc}

\pagestyle{fancy}
\fancyhf{} % clear all header and footer fields
\fancyfoot{}
\renewcommand{\headrulewidth}{0pt}
\renewcommand{\footrulewidth}{0pt}

% Adjust margins
\addtolength{\oddsidemargin}{-0.5in}
\addtolength{\evensidemargin}{-0.5in}
\addtolength{\textwidth}{1.0in}
\addtolength{\topmargin}{-.5in}
\addtolength{\textheight}{1.0in}

\urlstyle{same}

\newcommand{\blackHref}[2]{\href{#1}{\color{black}{#2}}}

\hypersetup{
  colorlinks=true,
  urlcolor=lightblue,
}

\raggedbottom
\raggedright
\setlength{\tabcolsep}{0in}

% Sections formatting
\titleformat{\section}{
  \vspace{-4pt}\scshape\raggedright\large
}{}{0em}{}[\color{black}\titlerule \vspace{-5pt}]

% -------------------------
% Custom commands
\newcommand{\resumeItem}[2]{
\vspace{-3pt}
\item\small{
    \textbf{#1}{\textbf{:} #2 \vspace{-2pt}}
  }
}

\newcommand{\resumeItemNoTitle}[1]{
\item\small{
    {#1 \vspace{-2pt}}
  }
}

\newcommand{\resumeSubheading}[4]{
  \vspace{4pt}
  \begin{tabular*}{\textwidth}{l@{\extracolsep{\fill}}r}
    \textbf{#1} & #2 \\
    {\small#3} & {\small #4} \\
  \end{tabular*}\vspace{-5pt}
}

\newcommand{\resumeSubItem}[2]{\resumeItem{#1}{#2}\vspace{-4pt}}

\renewcommand{\labelitemii}{$\circ$}

\newcommand{\resumeSubHeadingListStart}{}
\newcommand{\resumeSubHeadingListEnd}{}
\newcommand{\resumeItemListStart}{\vspace{3pt}\begin{itemize}}
  \newcommand{\resumeItemListEnd}{\end{itemize}\vspace{-5pt}}

\def\ci#1{\textcircled{\resizebox{.5em}{!}{#1}}}
\definecolor{lightblue}{RGB}{60, 200, 225}

\newcommand{\fakesection}[1]{%
  \par\refstepcounter{section}% Increase section counter
  \sectionmark{#1}% Add section mark (header)
  \addcontentsline{toc}{section}{\protect\numberline{\thesection}#1}% Add section to ToC
  % Add more content here, if needed.
}

% -------------------------------------------
%%%%%% CV STARTS HERE  %%%%%%%%%%%%%%%%%%%%%%%%%%%%


\begin{document}

% ----------HEADING-----------------
\begin{tabularx}{\textwidth}{ X r }
  {\vspace{-32pt}\color{lightblue}{\fontsize{45}{30}\selectfont Sasha Krassovsky}} &
  \begin{tabular}{ r }
    \ci{\faPhone}+1 (425) 614-9499 \\
    \blackHref{https://www.github.com/save-buffer}{\ci{\faGithub}save-buffer} \\
    \blackHref{mailto:krassovskysasha@gmail.com}{\ci{\faEnvelope}krassovskysasha@gmail.com} \\
    \blackHref{https://www.linkedin.com/in/sashka}{\ci{\faLinkedin}linkedin.com/in/sashka} \\
    \\
    \\
  \end{tabular}
\end{tabularx}

\fakesection{Goal}
\begin{tcolorbox}
I'm an engineer passionate in designing and applying novel approaches to make programs run
faster. I am interested in low-level optimizations, parallel computing, heterogeneous computing,
compilers, and distributed systems.
\end{tcolorbox}

\columnratio{0.25}
\begin{paracol}{2}

\begin{tcolorbox}[size=small]
\textbf{Projects}
\begin{itemize}[leftmargin=*]
\item \href{https://github.com/save-buffer/gigagrad}{\textbf{Gigagrad}}: A deep learning compiler
  in C++. Performs autodifferentiation and compiles to various backends.
\item \textbf{Steamed Hams}: A work-stealing, page-stealing, IO-completion stealing parallel runtime
  written in C. Will eventually have a SQL query engine on it.
\end{itemize}

\textbf{Achievements}
\begin{itemize}[leftmargin=*]
\item \href{https://x.com/AGIHouseSF/status/1793506669328031907}{\textbf{AGI House Hackathon Winner}}: Won
  the Hardcore Systems Hackathon by implementing a new backend for Gigagrad.
\end{itemize}
\vfill
\end{tcolorbox}
\switchcolumn[1]
\begin{tcolorbox}[breakable]
\fakesection{Experience}
\resumeSubHeadingListStart
\resumeSubheading{Neon}{Remote (Seattle, WA)}
{Systems Software Engineer}{March 2023 - Present}
\resumeItemListStart
\resumeItem{Network Monitoring}
{Implemented an eBPF-based ingress/egress network monitoring solution for Neon's VM infrastructure.}
\resumeItem{DiskANN}
{Prototyped a high performance DiskANN implementation with the eventual goal of getting put into
  a Postgres extension. Outperformed libhnsw's index build by over 2x}
\resumeItem{Role Management}
{Implemented Neon's role management, from synchronization between compute and control plane
  to the relevant UI changes.}
\resumeItemListEnd
\vspace{2pt}
\resumeSubheading{Voltron Data}{Remote (Seattle, WA)}
{Software Engineer}{October 2021 - March 2023}
\resumeItemListStart
\resumeItem{Hash Join Spilling}
{Designed and implemented a solution to joining larger-than-memory datasets by partitioning
  and writing the dataset to disk. Optimized for modern NVME SSDs, achieving maximum throughput.}
\resumeItem{Bloom Filter Pushdown}
{Implemented a novel dynamic Bloom filter strategy, where the filter is pushed as early as possible
  in a tree of joins, resulting in orders-of-magnitude speedups in certain workloads.}
\resumeItem{Performance Investigations}
{Investigated performance of various subcomponents of the query engine and prototyped solutions
  to performance issues.}
\resumeItemListEnd
\vspace{2pt}
\resumeSubheading{SingleStore}{Seattle, WA}
{Software Engineer: Query Execution}{June 2020 - October 2021}
\resumeItemListStart
\resumeItem{AVX-512 Prototype}
{Designed and implemented a prototype of AVX-512-based query execution in SingleStore, a
  natively-distributed database engine used for both transactions and analytics. The prototype
  achieved up to 20\% improvement on TPCH benchmark queries. Details regarding my experience
  can be found on my
  \href{https://www.singlestore.com/blog/a-programmers-perspective/}{blog post}.}
\resumeItem{GPU-based Query Execution}
{Proposed and drafted a design for GPU-based query execution, winning an internal contest
  for so-called ``moonshot ideas''. Prototyped aggregation on GPUs in an internal hackathon
  achieving a 2.5x speedup over CPU execution, demonstrating the promise of this approach.}
\resumeItem{Compiler Upgrade}
{Resolved performance regressions from upgrading the compiler. Compared assembly code between old and
  new compilers, and provided compiler hints to fix performance regressions.}
\resumeItem{Columnstore Query Execution}
{Proposed, designed, and implemented a significant refactor of core columnstore code,
  exposing several inefficiencies leading to a 2x speedup on some TPC-H queries.}
\resumeItem{Reading Group Lead}
{Led discussions of cutting-edge academic papers on a variety of topics including GPU-based
  query execution, learned indices, and E-graphs.}
\resumeItem{Intern Mentor}
{Mentored a summer intern who developed heuristics for various runtime decisions during
  query execution.}
\resumeItem{Interviewer}
{Interviewed intern and new-grad candidates.}
\resumeItemListEnd
\vspace{2pt}
\resumeSubheading
{Facebook}{Menlo Park, CA}
{Intern: Oculus Application Platform Team}{June 2019 - September 2019}
\vspace{2pt}
\resumeItemListStart
\resumeItem{Swift Playgrounds on Windows}
{Implemented an interactive programming environment on Windows mimicking Apple's
  Playgrounds. The program compiles and executes the input Swift code, and exposes an
  interface to develop UIs. Submitted patches to both the Swift compiler
  and LLVM. Source Code at \href{https://www.github.com/save-buffer/swift-repl}
  {github.com/save-buffer/swift-repl}.}
\resumeItemListEnd
\vspace{2pt}
\resumeSubheading
{Bespoke Silicon Group}{Seattle, WA}
{Undergraduate Researcher}{March 2019 - June 2020}
\resumeItemListStart
\vspace{2pt}
\resumeItem{Hammerblade Manycore}
{Created APIs for and tested a RISC-V Manycore CPU. Further, created highly optimized kernels
  for machine learning on the Manycore. Worked on compiler optimization to issue remote
  loads earlier.}
\resumeItem{Thesis: Exploring Single-Core Optimizations for Manycore Architectures}{Explored
  how I optimized kernels and details regarding my compiler optimization.}
\resumeItemListEnd
\vspace{2pt}
\resumeSubheading
{Husky Robotics}{Seattle, WA}
{Software Team Lead}{June 2018 - October 2019}
\vspace{2pt}
\resumeItemListStart
\resumeItem{Husky Robotics Software Team Lead}
{Helped design and implement the software for a mars rover. Implemented a wide range of
  software from sensors to networking to inverse kinematics to computer vision.  The team
  won 2nd place at the 2019 Canadian Rover Challenge.}
\resumeItemListEnd
\vspace{2pt}
\resumeSubheading
{MemSQL (now known as SingleStore)}{Seattle, WA}
{Software Engineering Intern}{Summer 2017 and 2018}
\resumeItemListStart
\resumeItem{Graphical Explain}
{Created a browser-based visualizer of query plans generated by MemSQL, aiding
  engineers in understanding query plans and its bottlenecks, ultimately allowing for
  more efficient query tuning.}
\resumeItem{Vectorized Aggregation}
{Designed and implemented algorithms for vectorized summation using AVX2. Achieved 1.8x
  speedup of queries involving summations, improving runtime of internal components by up
  to 20x.}
\resumeItem{Hash Join Optimization}
{Implemented an optimization to improve performance of hash joins involving integer keys
  by 2x, a key usecase for star-schema workloads.}
\resumeItemListEnd
\resumeSubheading
{Microsoft}{Redmond, WA}
{High School Intern}{June 2016 - August 2016}
\resumeSubHeadingListEnd

\fakesection{Education}
\resumeSubHeadingListStart
\vspace{8pt}
\begin{tabular*}{0.97\textwidth}{@{\extracolsep{\fill}}lrr}
  \textbf{University of Washington} & Overall GPA: 3.70 & Seattle, WA \\
  {\small BS Computer Science with Honors} & {\small In-Major GPA: 3.76} & {\small June 2020} \\
  {\small BS Discrete Mathematics} & {\small In-Major GPA: 3.62} & {\small June 2020} \\
\end{tabular*}\vspace{-5pt}

\resumeSubHeadingListEnd
\end{tcolorbox}
\end{paracol}

\end{document}
