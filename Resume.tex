\documentclass[letterpaper,11pt]{article}

\usepackage{latexsym}
\usepackage[empty]{fullpage}
\usepackage{titlesec}
\usepackage{marvosym}
\usepackage[usenames,dvipsnames]{color}
\usepackage{verbatim}
\usepackage{enumitem}
\usepackage{hyperref}
\usepackage{fancyhdr}
\usepackage{tabularx}
\usepackage{fontawesome}
\usepackage{tikz}
\usepackage{graphicx}

\usepackage[sfdefault, thin]{roboto}
\renewcommand\familydefault{\sfdefault} 
\usepackage[T1]{fontenc}

\pagestyle{fancy}
\fancyhf{} % clear all header and footer fields
\fancyfoot{}
\renewcommand{\headrulewidth}{0pt}
\renewcommand{\footrulewidth}{0pt}

% Adjust margins
\addtolength{\oddsidemargin}{-0.375in}
\addtolength{\evensidemargin}{-0.375in}
\addtolength{\textwidth}{1.0in}
\addtolength{\topmargin}{-.5in}
\addtolength{\textheight}{1.0in}

\urlstyle{same}

\raggedbottom
\raggedright
\setlength{\tabcolsep}{0in}

% Sections formatting
\titleformat{\section}{
  \vspace{-4pt}\scshape\raggedright\large
}{}{0em}{}[\color{black}\titlerule \vspace{-5pt}]

% -------------------------
% Custom commands
\newcommand{\resumeItem}[2]{
\item\small{
    \textbf{#1}{\textbf{:} #2 \vspace{-2pt}}
  }
}

\newcommand{\resumeItemNoTitle}[1]{
\item\small{
    {#1 \vspace{-2pt}}
  }
}

\newcommand{\resumeSubheading}[4]{
  \vspace{8pt}
  \begin{tabular*}{0.97\textwidth}{l@{\extracolsep{\fill}}r}
    \textbf{#1} & #2 \\
    {\small#3} & {\small #4} \\
  \end{tabular*}\vspace{-5pt}
}

\newcommand{\resumeSubItem}[2]{\resumeItem{#1}{#2}\vspace{-4pt}}

\renewcommand{\labelitemii}{$\circ$}

\newcommand{\resumeSubHeadingListStart}{}
\newcommand{\resumeSubHeadingListEnd}{}
\newcommand{\resumeItemListStart}{\begin{itemize}}
  \newcommand{\resumeItemListEnd}{\end{itemize}\vspace{-5pt}}

\def\ci#1{\textcircled{\resizebox{.5em}{!}{#1}}}
\definecolor{lightblue}{RGB}{60, 200, 225}

\newcommand{\fakesection}[1]{%
  \par\refstepcounter{section}% Increase section counter
  \sectionmark{#1}% Add section mark (header)
  \addcontentsline{toc}{section}{\protect\numberline{\thesection}#1}% Add section to ToC
  % Add more content here, if needed.
}

% -------------------------------------------
%%%%%% CV STARTS HERE  %%%%%%%%%%%%%%%%%%%%%%%%%%%%


\begin{document}

% ----------HEADING-----------------
\begin{tabularx}{\textwidth}{ X r }
  {\vspace{-32pt}\color{lightblue}{\fontsize{45}{30}\selectfont Sasha Krassovsky}} &
  \begin{tabular}{ r }
    \ci{\faPhone}+1 (425) 614-9499 \\
    \href{https://www.github.com/save-buffer}{\ci{\faGithub}save-buffer} \\
    \href{mailto:krassovskysasha@gmail.com}{\ci{\faEnvelope}krassovskysasha@gmail.com}\\
    \href{https://www.linkedin.com/in/sashka}{\ci{\faLinkedin}linkedin.com/in/sashka}\\
    \\
    \\
  \end{tabular}
\end{tabularx}

\fakesection{Skills}
\begin{tabularx}{\textwidth}{@{\extracolsep{\fill} } c c c c}
  Algorithms & SIMD & Compilers & Databases \\
  Embedded Systems & Code Optimization & OpenCV & Machine Learning \\
\end{tabularx}
\fakesection{Experience}
\resumeSubHeadingListStart

\vspace{4pt}
\resumeSubheading
{Facebook}{Menlo Park, CA}
{Software Engineering Intern: Oculus Application Platform Team}{June 2019 - Present}
\vspace{8pt}
\resumeItemListStart
\resumeItem{Swift Playgrounds on Windows}
{Implemented Swift Playgrounds on Windows, using GDI32 for UI, Apple's Swift compiler
  for frontend, and LLVM's ORC to JIT and execute the code. Involved modifications
  at both the AST level and the IR level, and submitting patches to both the compiler
  and LLVM. Source Code at \href{https://www.github.com/save-buffer/swift-repl}
  {github.com/save-buffer/swift-repl}}
\resumeItemListEnd
\vspace{4pt}
\resumeSubheading
{Bespoke Silicon Group}{Seattle, WA}
{Undergraduate Researcher}{March 2019 - Present}
\resumeItemListStart
\resumeItem{Hammberblade Manycore}
{Created APIs for and tested a RISC-V Manycore CPU. Further, created highly optimized kernels
  for machine learning on the Manycore.}
\resumeItemListEnd
\resumeSubheading
{MemSQL}{Seattle, WA}
{Software Engineering Intern: Query Execution}{June 2018 - September 2018}
\vspace{8pt}
\resumeItemListStart
\resumeItem{SIMD for Query Execution}
{Worked on MemSQL columnstore: Implemented low-level optimizations for summation in
  query execution engine with AVX2. Achieved up to 20x speedup of internal components,
  resulting in up to 1.8x speedup of total query execution time of customer workloads.}
\resumeItemListEnd
\vspace{4pt}
\resumeSubheading
{Husky Robotics}{Seattle, WA}
{Software Team Lead}{June 2018 - Present}
\vspace{8pt}
\resumeItemListStart
\resumeItem{Husky Robotics Software Team}
{The team works to create a Mars Rover to compete in the University and Canadian
  Rover Challenges. Wrote code ranging from sensor interfaces to networking to
  inverse kinematics to computer vision, and guided others in helping implement
  these things. Was a member of the team for one year prior to election as team
  leader. The team won 2nd place at the 2019 Canadian Rover Challenge.}
\resumeItemListEnd
\vspace{4pt}
\resumeSubheading
{MemSQL}{Seattle, WA}
{Software Engineering Intern}{August 2017 - September 2017}
\vspace{8pt}
\resumeItemListStart
\resumeItem{Hash Join Optimization}
{Worked on MemSQL in-memory distributed DBMS, optimizing Hash Join when joining on
  Integer Keys. Achieved ~2x speedup on Hash Join}
\resumeItem{Graphical Explain Plan}
{Created graphical explain plan visualizer for viewing a breakdown of queries.}
\resumeItemListEnd
\vspace{4pt}
\resumeSubheading
{Microsoft}{Redmond, WA}
{High School Intern}{June 2016 - August 2016}
\vspace{8pt}
\resumeItemListStart
\resumeItem{Cyberattack Analytics Dashboard}
{Developed a Web Application which visualized statistics on cyberattacks.
  On display in the CDOC (Cyber-Defense Operations Center)}
\resumeItem{HoloFlight}
{Participated in the OneWeek hackathon, where I worked on HoloFlight, an
  application for the Hololens that directs a drone to fly to waypoints selected
  on a holographic map. Won 2nd place in the HoloHack division.}
\resumeItemListEnd
\resumeSubHeadingListEnd

\fakesection{Education}
\resumeSubHeadingListStart
\vspace{8pt}
\begin{tabular*}{0.97\textwidth}{@{\extracolsep{\fill}}lrr}
  \textbf{University of Washington} & Overall GPA: 3.69 & Seattle, WA \\
  {\small BS Computer Science} & {\small In-Major GPA: 3.79} & {\small June 2020} \\
  {\small BS Discrete Mathematics} & {\small In-Major GPA: 3.63} & {\small June 2020} \\
\end{tabular*}\vspace{-5pt}

\resumeSubHeadingListEnd

\pagebreak

\fakesection{Coursework}
\resumeSubHeadingListStart

\resumeSubheading
{\color{lightblue}Coursework}{}
{}{}
\vspace{-8pt}
\resumeItemListStart
\resumeItem{Algorithms}
{Proof-based class focusing on using existing techniques to create and prove
new algorithms.}
\resumeItem{Digital Circuit Design}
{Project-based clsas focusing on introducing the basics of Verilog and FPGAs. Created a
  clone of Dance Dance Revolution on an FPGA using an LED matrix and four pushbuttons.}
\resumeItem{Deep Learning on Coursera}
{Introductory online machine learning course taught by Andrew Ng. Homework done in Python,
  with a combination of hand-written and premade machine learning software.}
\resumeItem{Embedded Systems}
{Project-based class focusing on the creation of an embedded operating system. Created
  a medical device using Arduino running a custom scheduler and task queue system.}
\resumeItem{Stanford Compilers on Coursera}
{Introductory online compilers course taught by Prof. Alex Aiken. Implemented a compiler
  for COOL (Classroom Object-Oriented Language) targetting MIPS.}
\resumeItem{3D Graphics at DigiPen Institute of Technology}
{Summer class introducing 3D Graphics, first algebraically and then using linear algebra.
  Implemented a 3D Graphics pipeline from scratch in C (starting only with a SetPixel
  function) which draws colorful cubes, and allows the user to rotate the camera.}
\resumeItemListEnd
\resumeSubHeadingListEnd

\fakesection{Accomplishments}
\resumeSubHeadingListStart

\resumeSubheading
{\color{lightblue}Accomplishments}{}
{}{}
\vspace{-8pt}
\resumeItemListStart
\resumeItem{2017 ACM ICPC Regionals Qualifier}
{Won 6th place at the UW ACM ICPC competition on team MATLAB Indexers, qualifying
  for regionals.}
\resumeItem{Hunt the Wumpus Alumni Challenge 2016 Winner}
{Programmed a game in C\# from scratch and was selected by a panel of judges as the winner for
  having the most polished and most entertaining game. Available at
  \href{https://www.github.com/save-buffer/ShooterGame}{github.com/save-buffer/ShooterGame}}
\resumeItemListEnd
\resumeSubHeadingListEnd


\fakesection{Projects}
\resumeSubHeadingListStart

\resumeSubheading
{\color{lightblue}Projects}{}
{}{}
\vspace{-8pt}
\resumeItemListStart
\resumeItem{Robotics Algorithms}
{Computer vision and inverse kinematics algorithms written for use by the Husky Robotics
  team. Computer vision can locate tennis balls in an image and segment a keyboard into
  keys. The inverse kinematics algorithm was created by myself and another member of the
  team and implemented by me. Available at
  \href{https://github.com/save-buffer/robot\_cv}{github.com/save-buffer/robot\_cv},p
  \href{https://github.com/save-buffer/UseArm}{github.com/save-buffer/UseArm}}
\resumeItem{Slang}
{Compiler for a low-level functional declarative language (like Erlang) written in C.
Available at \href{https://github.com/save-buffer/slang}{github.com/save-buffer/slang}}
\resumeItem{SashCopter}
{An attempt at controlling a drone using an EEG Headset. Main drone control software is
  written. Mind-controlled aspect deemed unsuccessful due to the lack of resolution on
  the headset.  
  \href{https://www.github.com/save-buffer/SashCopter}{github.com/save-buffer/SashCopter}}
\resumeItemListEnd
\resumeSubHeadingListEnd

\end{document}
